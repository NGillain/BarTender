%% Le document à compléter en LaTeX
%


%

% !TEX TS-program = pdflatex
% !TEX encoding = UTF-8 Unicode

% % !TEX %root


\documentclass[11pt, a4, oneside, headings=normal]{scrreprt}

\usepackage[utf8]{inputenc}
\usepackage[T1]{fontenc}
\usepackage{lmodern}
\usepackage[np]{numprint} % affichage correct des nombres : \numprint{} > \np{}
\usepackage{xspace}
\usepackage{graphicx}
\usepackage{color}
\usepackage{xcolor}
\usepackage{microtype}
\usepackage{textcomp}
\usepackage[hidelinks]{hyperref}
\usepackage{longtable}

\usepackage{booktabs} % for much better looking tables
\usepackage{array} % for better arrays (eg matrices) in maths
\usepackage{paralist} % very flexible & customisable lists (eg. enumerate/itemize, etc.)
\usepackage{verbatim} % adds environment for commenting out blocks of text & for better verbatim
\usepackage{subfig} % make it possible to include more than one captioned figure/table in a single float

\usepackage{amsmath}
\usepackage{listings}

\definecolor{mygreen}{rgb}{0,0.6,0}
\definecolor{mygray}{rgb}{0.5,0.5,0.5}
\definecolor{mymauve}{rgb}{0.58,0,0.82}

\lstset{ %
backgroundcolor=\color{white}, % choose the background color; you must add \usepackage{color} or \usepackage{xcolor}
basicstyle=\footnotesize, % the size of the fonts that are used for the code
breakatwhitespace=false, % sets if automatic breaks should only happen at whitespace
breaklines=true, % sets automatic line breaking
captionpos=b, % sets the caption-position to bottom
commentstyle=\color{mygreen}, % comment style
%deletekeywords={...}, % if you want to delete keywords from the given language
%escapeinside={\%*}{*)}, % if you want to add LaTeX within your code
%extendedchars=true, % lets you use non-ASCII characters; for 8-bits encodings only, does not work with UTF-8
frame=single, % adds a frame around the code
keepspaces=true, % keeps spaces in text, useful for keeping indentation of code (possibly needs columns=flexible)
keywordstyle=\color{blue}, % keyword style
language=Java, % the language of the code
%morekeywords={*,...}, % if you want to add more keywords to the set
numbers=left, % where to put the line-numbers; possible values are (none, left, right)
numbersep=5pt, % how far the line-numbers are from the code
numberstyle=\tiny\color{mygray}, % the style that is used for the line-numbers
rulecolor=\color{black}, % if not set, the frame-color may be changed on line-breaks within not-black text (e.g. comments (green here))
showspaces=false, % show spaces everywhere adding particular underscores; it overrides 'showstringspaces'
showstringspaces=true, % underline spaces within strings only
showtabs=false, % show tabs within strings adding particular underscores
stepnumber=5, % the step between two line-numbers. If it's 1, each line will be numbered
firstnumber=0,
stringstyle=\color{mymauve}, % string literal style
tabsize=4, % sets default tabsize to 2 spaces
% title=\lstname % show the filename of files included with \lstinputlisting; also try caption instead of title
}

\newcommand{\javaprogram}[1]{\subsection{\texttt{#1}}\lstinputlisting{RATTA/#1}}

\renewcommand{\arraystretch}{1,2}

\newcommand{\HRule}{\rule{\linewidth}{0.5mm}}
\usepackage{tabu,enumerate}

\usepackage[frenchb]{babel}

\setlength{\parskip}{0.35cm}

\begin{document}
%\maketitle

\begin{titlepage}
\begin{center}

\includegraphics[width=0.25\textwidth]{epl-logo}~\\[0.4cm]

\textsc{\LARGE École Polytechnique de Louvain}\\[1.2cm]
\textsc{\Large [LSINF~1225]}\\[0.05cm]
\textsc{\Large Conception Orientée Objet }\\[0.1cm]
\textsc{\Large et Gestion de Données}\\[1.0cm]


\textsc{\Large Groupe F --- Année 2014--2015}\\[0.8cm]

% Title
\HRule \\[0.5cm]
\textsc{\LARGE  Travail 1}\\[0.1cm]
{ \huge \bfseries La Modélisation de Données \\[0.3cm] }

\HRule \\[0.6cm]

% Authors, professors, tutor
{\large
\begin{tabu} to 0.7\linewidth {Xll}
    \emph{Auteurs:}\\
    \quad Mathieu \textsc{Delandmeter} & 6240\,--\,13\,--\,00\\
    \quad Nathan \textsc{Gillain} & 7879\,--\,12\,--\,00\\
    \quad Maxime \textsc{Hanot} & 6591\,--\,13\,--\,00\\
    \quad Alexandre \textsc{Jadin} & 4844\,--\,13\,--\,00\\
    \quad Thomas \textsc{Marissal} & 8217\,--\,13\,--\,00\\
    \quad Edouard \textsc{Vangangel} & 2243\,--\,09\,--\,00\\[.5ex]   
    \emph{Professeur:} \\
    \quad Kim \textsc{Mens}\\[.5ex]
    \emph{Tuteur:}\\
    \quad Benoît \textsc{Baufays}\\
\end{tabu}
}

\vfill

% Bottom of the page
{\large \today}

\end{center}
\end{titlepage}

%%%%%% LE DOCUMENT COMMENCE ICI %%%%%%%%

%\tableofcontents

\subsection*{Introduction}

Dans le cadre du cours de conception orientée objet et gestion de données, dispensé par le Professeur Kim Mens, il nous a été demandé d'implémenter une application \textit{Android} de type \textit{"Gestion de bar"}. Dans un premier temps, nous avons dû élaborer une base de donnée avec l'outil SQLite, tâche qui a nécessité plusieurs étapes. Dès lors, ce document, premier rapport de ce projet, a pour but d'expliquer la démarche que nous avons suivie ainsi que les délivrables que nous avons créés.

\subsection*{Démarche}

Tout d'abord, nous avons dû déterminer les faits élémentaires qui étaient nécessaires à la construction de notre base de données. Ces faits élémentaires se retrouvent dans le fichier \textit{"faitselementaires.pdf"}. Ce fichier intègre en même temps une "population" afin de donner des exemples pour chacune des différentes données.

A ce stade du projet, nous avons dû choisir les extensions que nous implémenterons, la description de celles-ci se trouve dans la suite de ce rapport, dans le section "Extensions".

Ensuite, grâce au logiciel \textit{"Dia"}, nous avons créé un schéma conceptuel ORM (\textit{DiagrammeORM\_GroupeF.png}) afin de visualiser les liens entre les différentes entités de notre système. Nous avons également dû indiquer sur ce schéma les contraintes d'unicité ainsi que les rôles qui étaient obligatoires.

Par la suite, nous avons traduit ce schéma conceptuel ORM en un schéma relationnel se trouvant dans le fichier "Schéma\_relationnel.png". Cette étape a permis de rendre beaucoup plus simple la création de notre base de donnée.

Enfin, nous avons encore réalisé deux étapes. La première consiste en la création de la base de données proprement dite (fichier Bartender.sqlite) tandis que la seconde est la création d'une liste reprenant toutes les commandes qui nous ont permis de tester si notre base de donnée avait été correctement construite. Cette dernière se trouve dans ce rapport dans la section "Requête SQL".

\subsection*{Extensions}

\paragraph{Première extension : "Historique"}
L'utilisateur de l'application aura l'occasion de consulter un historique, différent selon son statut :
\begin{itemize}
\item Le client pourra voir quels achats il a effectués lors du mois écoulé, lui permettant ainsi de savoir à quelle fréquence il se rend dans le bar, qu'est-ce qu'il y consomme, combien il y consomme, lui permettant même ainsi de pouvoir faire attention à sa consommation d'alcool ou, pourquoi pas, à l'inverse (mais cela n'est éthiquement pas recommandé), de faire des concours avec ses amis pour qui boira le plus.

\item Le patron pourra quant à lui consulter un historique de l'ensemble des achats effectués dans son bar, et cela lui servira de base à des statistiques très utiles sur les habitudes de ses clients, lui permettant de prévoir les stocks à avoir, comment adapter les prix, pour quels produits effectuer des promotions, etc.
\end{itemize}

\paragraph{Deuxième extension : "Préférences"}
Cette extension va de paire avec l'historique : en effet, le client aura des préférences explicites et implicites.
\begin{itemize}
\item Explicites : le client pourra indiquer dans son profil son âge, sa religion et ses allergies, ce qui permettra à l'application de ne pas lui proposer certains produits proscrits.
\item Implicites : grâce à l'historique qui permettra à l'application de savoir quels produits sont généralement consommés par le client, celle-ci pourra adapter l'ordre de la carte pour lui proposer en premier lieu les boissons préférées du client.
\end{itemize}

\subsection*{Requêtes SQL}

\begin{itemize}
\item Compter combien de types de boissons différentes on peut trouver sur la carte de boissons.
\\ Select Count(Nom) FROM Boisson
\item Trouver toutes les boissons qui font partie d’une commande donnée.
 \\ Select Boisson FROM Consommation WHERE AddNum=1
\item Calculer le total pour une commande donnée.
 \\ Select sum(B.PRIXVENTE*C.Qté) FROM Consommation C, Boisson B, Addition A WHERE A.AddNum=1 AND A.AddNum=C.AddNum AND B.Nom=C.Boisson
\item Trouver toutes les boissons, et leur nombre, vendues par un serveur donné.
 \\ Select Con.nom, SUM(Con.nboisson) FROM addition Add, consommation Con WHERE Add.login\_serveur = "Serveur Donné" AND \\ Con.IDaddition = Add.IDaddition GROUP BY Con.nom
\item Calculer l’addition pour une table donnée, sachant que cette table peut avoir fait plusieurs commandes.
 \\ Select sum(C.Qté) FROM Boisson B, Consommation C, Addition A, Utilisateur U WHERE U.Login='AMaalouf' AND U.Login=A.ServeurLogin AND \\ A.Num=C.AddNum AND C.Boisson=B.NOM GROUP BY B.NOM
\item Trouver toutes les boissons dont ils ne restent plus assez en stock (qui sont en dessous du seuil et
qui doivent donc être commandées chez le fournisseur).
 \\ Select B.Nom FROM Boisson B WHERE B.Stock<B.Seuil
\item Trouver toutes les boissons contenant un allergène du Client
 \\ Select Ai.Login,Ae.NomBoisson FROM Allergene Ae,Allergies Ai WHERE \\ Ae.Allergene=Ai.Allergene AND Ae.Login=Abra
\item Trouver toutes les boissons qui ne contiennent aucun allergène d'un client donné par son login
 \\ Select B.nom FROM Boisson B, Allergies\_boisson Ab, Allergies A WHERE B.nom = Ab.nom AND A.login = Abra AND A.allergène <> Ab.allergène

\end{itemize}


\end{document}