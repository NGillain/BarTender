%% Le document à compléter en LaTeX
% 


%

% !TEX TS-program = pdflatex
% !TEX encoding = UTF-8 Unicode

% % !TEX %root


\documentclass[10pt, a4, oneside, headings=normal]{scrartcl}

\usepackage{url}


\usepackage{mhchem}

\usepackage[utf8]{inputenc}
\usepackage[T1]{fontenc}
\usepackage{lmodern}
\usepackage[np]{numprint} % affichage correct des nombres : \numprint{} > \np{}
\usepackage{xspace}

\usepackage{booktabs} % for much better looking tables
\usepackage{array} % for better arrays (eg matrices) in maths
\usepackage{paralist} % very flexible & customisable lists (eg. enumerate/itemize, etc.)
\usepackage{verbatim} % adds environment for commenting out blocks of text & for better verbatim
\usepackage{subfig} % make it possible to include more than one captioned figure/table in a single float

\usepackage{amsmath}

\usepackage{graphicx}
\usepackage{SIUnits}

\usepackage{setspace} %pour la taille de l'interligne -> 1 par défaut
\setstretch{1}

\usepackage{geometry}
\geometry{vmargin=2cm}


\usepackage[frenchb]{babel}

\begin{document}

\titlehead{}
\subject{}
\title{Extensions}
\subtitle{}
\author{Groupe F}
\publishers{}
\date{\today}

\dedication{}

\maketitle

%%Ici commence le document

\section{Première extension : "Historique"}
L'utilisateur de l'application aura l'occasion de consulter un historique, différent selon son statut :
\begin{itemize}
\item Le client pourra voir quels achats il a effectués lors du mois écoulé, lui permettant ainsi de savoir à quelle fréquence il se rend dans le bar, qu'est-ce qu'il y consomme, combien il y consomme, lui permettant même ainsi de pouvoir faire attention à sa consommation d'alcool ou, pourquoi pas, à l'inverse (mais cela n'est éthiquement pas recommandé), de faire des concours avec ses amis pour qui boira le plus.

\item Le patron pourra quant à lui consulter un historique de l'ensemble des achats effectués dans son bar, et cela lui servira de base à des statistiques très utiles sur les habitudes de ses clients, lui permettant de prévoir les stocks à avoir, comment adapter les prix, pour quels produits effectuer des promotions, etc.
\end{itemize}

\section{Deuxième extension : "Préférences"}
Cette extension va de paire avec l'historique : en effet, le client aura des préférences explicites et implicites.
\begin{itemize}
\item Explicites : le client pourra indiquer dans son profil son âge, sa religion et ses allergies, ce qui permettra à l'application de ne pas lui proposer certains produits proscrits.
\item Implicites : grâce à l'historique qui permettra à l'application de savoir quels produits sont généralement consommés par le client, celle-ci pourra adapter l'ordre de la carte pour lui proposer en premier lieu les boissons préférées du client.
\end{itemize}


\end{document}
