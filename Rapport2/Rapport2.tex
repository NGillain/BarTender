%% Le document à compléter en LaTeX
%


%

% !TEX TS-program = pdflatex
% !TEX encoding = UTF-8 Unicode

% % !TEX %root


\documentclass[11pt, a4, oneside, headings=normal]{scrreprt}

\usepackage[utf8]{inputenc}
\usepackage[T1]{fontenc}
\usepackage{lmodern}
\usepackage[np]{numprint} % affichage correct des nombres : \numprint{} > \np{}
\usepackage{xspace}
\usepackage{graphicx}
\usepackage{color}
\usepackage{xcolor}
\usepackage{microtype}
\usepackage{textcomp}
\usepackage[hidelinks]{hyperref}
\usepackage{longtable}

\usepackage{booktabs} % for much better looking tables
\usepackage{array} % for better arrays (eg matrices) in maths
\usepackage{paralist} % very flexible & customisable lists (eg. enumerate/itemize, etc.)
\usepackage{verbatim} % adds environment for commenting out blocks of text & for better verbatim
\usepackage{subfig} % make it possible to include more than one captioned figure/table in a single float

\usepackage{amsmath}
\usepackage{listings}

\definecolor{mygreen}{rgb}{0,0.6,0}
\definecolor{mygray}{rgb}{0.5,0.5,0.5}
\definecolor{mymauve}{rgb}{0.58,0,0.82}

\lstset{ %
backgroundcolor=\color{white}, % choose the background color; you must add \usepackage{color} or \usepackage{xcolor}
basicstyle=\footnotesize, % the size of the fonts that are used for the code
breakatwhitespace=false, % sets if automatic breaks should only happen at whitespace
breaklines=true, % sets automatic line breaking
captionpos=b, % sets the caption-position to bottom
commentstyle=\color{mygreen}, % comment style
%deletekeywords={...}, % if you want to delete keywords from the given language
%escapeinside={\%*}{*)}, % if you want to add LaTeX within your code
%extendedchars=true, % lets you use non-ASCII characters; for 8-bits encodings only, does not work with UTF-8
frame=single, % adds a frame around the code
keepspaces=true, % keeps spaces in text, useful for keeping indentation of code (possibly needs columns=flexible)
keywordstyle=\color{blue}, % keyword style
language=Java, % the language of the code
%morekeywords={*,...}, % if you want to add more keywords to the set
numbers=left, % where to put the line-numbers; possible values are (none, left, right)
numbersep=5pt, % how far the line-numbers are from the code
numberstyle=\tiny\color{mygray}, % the style that is used for the line-numbers
rulecolor=\color{black}, % if not set, the frame-color may be changed on line-breaks within not-black text (e.g. comments (green here))
showspaces=false, % show spaces everywhere adding particular underscores; it overrides 'showstringspaces'
showstringspaces=true, % underline spaces within strings only
showtabs=false, % show tabs within strings adding particular underscores
stepnumber=5, % the step between two line-numbers. If it's 1, each line will be numbered
firstnumber=0,
stringstyle=\color{mymauve}, % string literal style
tabsize=4, % sets default tabsize to 2 spaces
% title=\lstname % show the filename of files included with \lstinputlisting; also try caption instead of title
}

\newcommand{\javaprogram}[1]{\subsection{\texttt{#1}}\lstinputlisting{RATTA/#1}}

\renewcommand{\arraystretch}{1,2}

\newcommand{\HRule}{\rule{\linewidth}{0.5mm}}
\usepackage{tabu,enumerate}

\usepackage[frenchb]{babel}

\setlength{\parskip}{0.35cm}

\begin{document}
%\maketitle

\begin{titlepage}
\begin{center}

\includegraphics[width=0.25\textwidth]{epl-logo}~\\[0.4cm]

\textsc{\LARGE École Polytechnique de Louvain}\\[1.2cm]
\textsc{\Large [LSINF~1225]}\\[0.05cm]
\textsc{\Large Conception Orientée Objet }\\[0.1cm]
\textsc{\Large et Gestion de Données}\\[1.0cm]


\textsc{\Large Groupe F --- Année 2014--2015}\\[0.8cm]

% Title
\HRule \\[0.5cm]
\textsc{\LARGE  Travail 2}\\[0.1cm]
{ \huge \bfseries La Conception Orientée Objet\\[0.3cm] }

\HRule \\[0.6cm]

% Authors, professors, tutor
{\large
\begin{tabu} to 0.7\linewidth {Xll}
    \emph{Auteurs:}\\
    \quad Mathieu \textsc{Delandmeter} & 6240\,--\,13\,--\,00\\
    \quad Nathan \textsc{Gillain} & 7879\,--\,12\,--\,00\\
    \quad Maxime \textsc{Hanot} & 6591\,--\,13\,--\,00\\
    \quad Alexandre \textsc{Jadin} & 4844\,--\,13\,--\,00\\
    \quad Thomas \textsc{Marissal} & 8217\,--\,13\,--\,00\\
    \quad Edouard \textsc{Vangangel} & 2243\,--\,09\,--\,00\\[.5ex]   
    \emph{Professeur:} \\
    \quad Kim \textsc{Mens}\\[.5ex]
    \emph{Tuteur:}\\
    \quad Benoît \textsc{Baufays}\\
\end{tabu}
}

\vfill

% Bottom of the page
{\large \today}

\end{center}
\end{titlepage}

%%%%%% LE DOCUMENT COMMENCE ICI %%%%%%%%

%\tableofcontents

\subsection*{Introduction}

Dans le cadre du cours de conception orientée objet et gestion de données, dispensé par le Professeur Kim Mens, il nous a été demandé d'implémenter une application \textit{Android} de type \textit{"Gestion de bar"}. Dans un premier temps, nous avons dû élaborer une base de donnée avec l'outil SQLite, tâche qui a nécessité plusieurs étapes. Dès lors, ce document, premier rapport de ce projet, a pour but d'expliquer la démarche que nous avons suivie ainsi que les délivrables que nous avons créés.

\subsection*{Description des documents rendus}

Cette section a pour but d'énumérer les différents documents que nous avons produit ainsi que d'en donner une courte description.

\paragraph*{Histoires d'utilisateur}

\paragraph*{Cartes CRC}



\paragraph*{Diagramme de classe UML conceptuel}

\paragraph*{Diagrammes de séquence UML}

Nous avons réalisé quatre diagrammes de séquences UML. Ces diagrammes correspondent aux quatre histoires d'utilisateur suivantes : 
\begin{itemize}
\item En tant que client, afin de consulter la carte, je dois créer un compte protégé par un mot de passe.
\item En tant que client, je dois pouvoir consulter la carte des boissons afin de voir la liste des boissons disponibles en tenant compte de mes préférences.
\item En tant que serveur, je veux pouvoir gérer les stocks afin de savoir combien de boissons sont encore en stock et donc savoir lesquelles doivent être commandées chez le fournisseur.
\item En tant que serveur, je dois pouvoir gérer la facturation afin de faire payer le client. (2)
\end{itemize}

\subsection*{Choix de conceptions}

\begin{itemize}
\item Concernant les cartes CRC, nous avions dans un premier temps créé une carte \textit{"Addition"}. Cependant lors de la réalisation du diagramme de classe UML, nous nous sommes rendus compte que celle-ci n'avait pas d'intérêt. En effet, la classe \textit{Commande} était suffisante.

\item

\end{itemize}


\end{document}