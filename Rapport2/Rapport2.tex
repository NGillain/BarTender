%% Le document à compléter en LaTeX
%


%

% !TEX TS-program = pdflatex
% !TEX encoding = UTF-8 Unicode

% % !TEX %root


\documentclass[11pt, a4, oneside, headings=normal]{scrreprt}

\usepackage[utf8]{inputenc}
\usepackage[T1]{fontenc}
\usepackage{lmodern}
\usepackage[np]{numprint} % affichage correct des nombres : \numprint{} > \np{}
\usepackage{xspace}
\usepackage{graphicx}
\usepackage{color}
\usepackage{xcolor}
\usepackage{microtype}
\usepackage{textcomp}
\usepackage[hidelinks]{hyperref}
\usepackage{longtable}

\usepackage{booktabs} % for much better looking tables
\usepackage{array} % for better arrays (eg matrices) in maths
\usepackage{paralist} % very flexible & customisable lists (eg. enumerate/itemize, etc.)
\usepackage{verbatim} % adds environment for commenting out blocks of text & for better verbatim
\usepackage{subfig} % make it possible to include more than one captioned figure/table in a single float

\usepackage{amsmath}
\usepackage{listings}

\definecolor{mygreen}{rgb}{0,0.6,0}
\definecolor{mygray}{rgb}{0.5,0.5,0.5}
\definecolor{mymauve}{rgb}{0.58,0,0.82}

\lstset{ %
backgroundcolor=\color{white}, % choose the background color; you must add \usepackage{color} or \usepackage{xcolor}
basicstyle=\footnotesize, % the size of the fonts that are used for the code
breakatwhitespace=false, % sets if automatic breaks should only happen at whitespace
breaklines=true, % sets automatic line breaking
captionpos=b, % sets the caption-position to bottom
commentstyle=\color{mygreen}, % comment style
%deletekeywords={...}, % if you want to delete keywords from the given language
%escapeinside={\%*}{*)}, % if you want to add LaTeX within your code
%extendedchars=true, % lets you use non-ASCII characters; for 8-bits encodings only, does not work with UTF-8
frame=single, % adds a frame around the code
keepspaces=true, % keeps spaces in text, useful for keeping indentation of code (possibly needs columns=flexible)
keywordstyle=\color{blue}, % keyword style
language=Java, % the language of the code
%morekeywords={*,...}, % if you want to add more keywords to the set
numbers=left, % where to put the line-numbers; possible values are (none, left, right)
numbersep=5pt, % how far the line-numbers are from the code
numberstyle=\tiny\color{mygray}, % the style that is used for the line-numbers
rulecolor=\color{black}, % if not set, the frame-color may be changed on line-breaks within not-black text (e.g. comments (green here))
showspaces=false, % show spaces everywhere adding particular underscores; it overrides 'showstringspaces'
showstringspaces=true, % underline spaces within strings only
showtabs=false, % show tabs within strings adding particular underscores
stepnumber=5, % the step between two line-numbers. If it's 1, each line will be numbered
firstnumber=0,
stringstyle=\color{mymauve}, % string literal style
tabsize=4, % sets default tabsize to 2 spaces
% title=\lstname % show the filename of files included with \lstinputlisting; also try caption instead of title
}

\newcommand{\javaprogram}[1]{\subsection{\texttt{#1}}\lstinputlisting{RATTA/#1}}

\renewcommand{\arraystretch}{1,2}

\newcommand{\HRule}{\rule{\linewidth}{0.5mm}}
\usepackage{tabu,enumerate}

\usepackage[frenchb]{babel}

\setlength{\parskip}{0.35cm}

\begin{document}
%\maketitle

\begin{titlepage}
\begin{center}

\includegraphics[width=0.25\textwidth]{epl-logo}~\\[0.4cm]

\textsc{\LARGE École Polytechnique de Louvain}\\[1.2cm]
\textsc{\Large [LSINF~1225]}\\[0.05cm]
\textsc{\Large Conception Orientée Objet }\\[0.1cm]
\textsc{\Large et Gestion de Données}\\[1.0cm]


\textsc{\Large Groupe F --- Année 2014--2015}\\[0.8cm]

% Title
\HRule \\[0.5cm]
\textsc{\LARGE  Travail 2}\\[0.1cm]
{ \huge \bfseries La Conception Orientée Objet\\[0.3cm] }

\HRule \\[0.6cm]

% Authors, professors, tutor
{\large
\begin{tabu} to 0.7\linewidth {Xll}
    \emph{Auteurs:}\\
    \quad Mathieu \textsc{Delandmeter} & 6240\,--\,13\,--\,00\\
    \quad Nathan \textsc{Gillain} & 7879\,--\,12\,--\,00\\
    \quad Maxime \textsc{Hanot} & 6591\,--\,13\,--\,00\\
    \quad Alexandre \textsc{Jadin} & 4844\,--\,13\,--\,00\\
    \quad Thomas \textsc{Marissal} & 8217\,--\,13\,--\,00\\
    \quad Edouard \textsc{Vangangel} & 2243\,--\,09\,--\,00\\[.5ex]   
    \emph{Professeur:} \\
    \quad Kim \textsc{Mens}\\[.5ex]
    \emph{Tuteur:}\\
    \quad Benoît \textsc{Baufays}\\
\end{tabu}
}

\vfill

% Bottom of the page
{\large \today}

\end{center}
\end{titlepage}

%%%%%% LE DOCUMENT COMMENCE ICI %%%%%%%%

%\tableofcontents

\section*{Introduction}

Dans le cadre du cours de conception orientée objet et gestion de données, dispensé par le Professeur Kim Mens, il nous a été demandé d'implémenter une application \textit{Android} de type \textit{"Gestion de bar"}. 

Après avoir élaboré une base de données en première partie de ce projet, nous nous sommes intéressés à la conception orientée objet.

Ce rapport a pour but de lister et de décrire les différents documents que nous avons créés ainsi que d'expliquer nos choix de conception.

Nous avons pour commencer créé des histoires d'utilisateurs et des cartes CRC, outils qui nous ont permis de créer un diagramme de classe UML et quelques diagrammes de séquences UML. Ces deux derniers documents seront essentiels à l'implémentation de notre application.

\section*{Description des documents rendus}

Cette section a pour but d'énumérer les différents documents que nous avons produits ainsi que d'en donner une courte description.

\subsection*{Histoires d'utilisateur}

Nous avons rédigé toute une liste d'histoires d'utilisateur. Nous les avons réparties en huit catégories :
\begin{itemize}
\item Gestion des utilisateurs
\item Sélectionner ses préférences
\item Afficher l'historique
\item Afficher la commande
\item Afficher la liste des boissons
\item Gestion des stocks
\item Gestion de la carte
\item Gestion des factures
\end{itemize}

\subsection*{Cartes CRC}

Nous avons réalisé huit cartes CRC. Celles-ci correspondent chacune à une des classes suivantes :

\begin{center}
\emph{Accueil, Utilisateur, Client, Serveur, Patron, Boisson, Commande, Addition}
\end{center}

Bien que celles-ci n'aient pas été mises à jour et donc que ce sont des brouillons, elles nous on permis de bien comprendre comment nous allions organiser notre code. Nous les avons dès lors utilisées afin de créer notre diagramme de classe UML.

\subsection*{Diagramme de classe UML}

A partir des cartes CRC, nous avons créé un diagramme de classe UML. Celui-ci modélise toutes les classes que nous implémenterons dans notre application Android. Il reprend les variables et fonctions principales de chaque classe ainsi que les liens entre ces dernières.

\subsection*{Diagrammes de séquences UML}

Nous avons réalisé quatre diagrammes de séquences UML. Ces diagrammes correspondent aux quatre histoires d'utilisateur suivantes : 

\begin{itemize}
\item En tant que client, afin de consulter la carte, je dois créer un compte protégé par un mot de passe.
\item En tant que client, je dois pouvoir consulter la carte des boissons (ainsi que d'effectuer une recherche sur base de mot-clés) afin de voir la liste des boissons disponibles en tenant compte de mes préférences.
\item En tant que serveur, je veux pouvoir gérer les stocks afin de savoir combien de boissons sont encore en stock et donc savoir lesquelles doivent être commandées chez le fournisseur.
\item En tant que serveur, je dois pouvoir gérer la facturation afin de faire payer le client.
\end{itemize}

\section*{Choix de conceptions}

\subsection*{Cartes CRC}

Concernant les cartes CRC, nous avions dans un premier temps créé une carte \textit{"Addition"}. Cependant lors de la réalisation du diagramme de classe UML, nous nous sommes rendus compte que celle-ci n'avait pas d'intérêt. En effet, la classe \textit{"Commande"} était suffisante pour gérer tous les cas.

A l'inverse, nous nous sommes rendus compte au même moment que pour pouvoir gérer notre extension "Préférence", nous avions besoin d'une classe \textit{"Historique"}. Cette dernière aura une fonction qui s'adapte à l'argument qu'on lui donne. En effet, si elle est appelée avec un argument de type "Client", l'historique des boissons qui ont été bues par ce client sera affiché. Inversement, si la fonction est appelée avec un argument de type "Boisson", elle affichera l'historique des clients qui ont bu cette boisson.

\subsection*{Diagramme de classe UML et diagrammes de séquences UML}

Dans le diagramme de classe UML, l'argument "photo" de boisson est de type string car il s'agit du chemin où se trouve la photo sur le dispositif.

Dans les diagrammes de séquences UML, un objet BDD (base de données) apparait alors qu'il n'y en a pas dans le diagramme de classe.
En effet, dans le diagramme de classe, on décrit la structure du programme sans tenir compte de l'aspect base de données tandis que dans les diagrammes de séquences il est beaucoup plus logique d'expliquer l'accès à celle-ci afin de bien comprendre d'où viennent les données.

\subsection*{Utilisation de la base de données}

Concernant l'utilisation de la base de données, nous avons décidé de ne pas charger l'ensemble de la base de données en instance Java. Nous nous connectons donc à celle-ci lorsque nous avons besoin d'un de ses éléments que nous transformons en instance d'objet Java. Quand nous n'avons plus besoin de celui-ci, nous le supprimons.

\end{document}
