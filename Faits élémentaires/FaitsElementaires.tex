%% Le document à compléter en LaTeX
% 


%

% !TEX TS-program = pdflatex
% !TEX encoding = UTF-8 Unicode

% % !TEX %root


\documentclass[10pt, a4, oneside, headings=normal]{scrartcl}

\usepackage{url}


\usepackage{mhchem}

\usepackage[utf8]{inputenc}
\usepackage[T1]{fontenc}
\usepackage{lmodern}
\usepackage[np]{numprint} % affichage correct des nombres : \numprint{} > \np{}
\usepackage{xspace}

\usepackage{booktabs} % for much better looking tables
\usepackage{array} % for better arrays (eg matrices) in maths
\usepackage{paralist} % very flexible & customisable lists (eg. enumerate/itemize, etc.)
\usepackage{verbatim} % adds environment for commenting out blocks of text & for better verbatim
\usepackage{subfig} % make it possible to include more than one captioned figure/table in a single float

\usepackage{amsmath}

\usepackage{graphicx}
\usepackage{SIunits}

\usepackage{setspace} %pour la taille de l'interligne -> 1 par défaut
\setstretch{1}

\usepackage{geometry}
\geometry{vmargin=2cm}


\usepackage[frenchb]{babel}

\begin{document}

\titlehead{}
\subject{}
\title{Faits élémentaires}
\subtitle{ }
\author{Groupe F}
\publishers{}
\date{\today}

\dedication{}

\maketitle

%%Ici commence le document
Un utilisateur (login) "Michmichdu32"
\begin{itemize}
\item a un nom (string) "Michel Jacques"
\item a un mot de passe (string) "iloveoz"
\item possède un statut (serveur, patron, client) (string) "client"
\item a un sexe (M, F) (string) "M"
\item a une religion (string) "Islam"
\item a un âge (numéro) 54
\item a des allergies "gluten"
\end{itemize}

Un utilisateur (login) "Richy"
\begin{itemize}
\item a un nom (string) "Richard Vaillant"
\item a un mot de passe (string) "HSBC2156"
\item possède un statut (serveur, patron, client) (string) "serveur"
\item a un sexe (M, F) (string) "M"
\end{itemize}

Une boisson (nom) "Café"
\begin{itemize}
\item a un prix d'achat (Euro) 0.3
\item a un prix de vente HTVA (Euro) 2
\item possède une description (string) "Expresso italien, doux et agréable"
\item a une photo (string) "expresso.png"
\item a un type (string) "Boisson chaude"
\item a une quantité en stock (numéro) 36
\item a un seuil (numéro) 24
\item a un maximum (numéro) 58
\item contient des allergènes (string) "lactose"
\item a une quantité (numéro) 2 pour une addition (numéro) 1
\end{itemize}

Un type (string) "Boisson chaude"
\begin{itemize}
\item a un logo (string) "boissonchaude.jpg"
\end{itemize}

Une addition (numéro) 1
\begin{itemize}
\item a une date (string) "2015-03-04"
\item possède une table (numéro) 1
\item correspond à un utilisateur comme client (login) "Michmichdu32"
\item correspond à un utilisateur comme serveur (login) "Richy"
\end{itemize}

\end{document}